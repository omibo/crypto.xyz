\chapter{Polynomial Functions}
\begin{definition}
    A \textit{polynomial function} $f$ of degree $d$ is a function of the form
    \[
        f(x) = c_0 + c_1 x + c_2 x^2 + \cdots + c_d x^d,
    \]
    where $c_d \neq 0$. Each term $c_i x^i$ is called a monomial.
\end{definition}

A polynomial function
\[
    f(X) \in \F^{(\leq d)}[X]
\]
is said to be of degree at most $d$, where the coefficients are taken from the finite field $\F$.

In such a polynomial, all arithmetic operations—such as addition and multiplication—are performed in $\F$. 
For example, to compute the expression $c_0 + c_2 x^2$, one first computes $x^2 = x \cdot x$ in $\F$, 
then multiplies by $c_2$, and finally adds $c_0$, with each operation carried out in $\F_p$.

\begin{remark}
    Polynomials that have no roots in the real numbers may possess roots in a finite field, 
    and conversely, polynomials that have real roots may have no roots in a finite field \cite{rareskills_finitefields}.
\end{remark}

\begin{definition}
    A \textit{multivariate polynomial function} $f(X_1, X_2, \ldots, X_n)$ is a polynomial function in more than one variable. 
    A polynomial function in a single variable is called \textit{univariate}.
\end{definition}

In a multivariate polynomial function with $\ell$ variables, each term(monomial) has the form 
\[
    c \, X_1^{d_1} X_2^{d_2} \cdots X_{\ell}^{d_{\ell}},
\]
and its degree is given by $d_1 + d_2 + \cdots + d_{\ell}$. 
The \textit{total degree} of the polynomial is the maximum degree among all its monomials. 
Multivariate polynomial functions over a filed $\F$ are commonly denoted either as $f(x_1, x_2, \ldots, x_{\ell})$, with each $x_i \in \F$, or as $f(x)$ where $x \in \F^{\ell}$.


\section{SZDL Lemma}
\begin{definition}
    In a polynomial function $f(X)$, an element $x$ is called a \textit{root} (or \textit{zero}) of $f$ if
    \[
        f(x) = 0.
    \]
\end{definition}

\begin{theorem}
    Let $f(X) \in \F^{(\leq d)}[X]$ be a polynomial of degree at most $d$ over the finite field $\F$. Then $f(X)$ has at most $d$ distinct roots.
\end{theorem}

\begin{proof}
    This an informal proof. Assume for the sake of contradiction that $f(X)$ has $d+1$ distinct roots, say $x_1, x_2, \ldots, x_{d+1}$. 
    Then $f(X)$ is divisible by
    \[
        (X - x_1)(X - x_2) \cdots (X - x_{d+1}),
    \]
    which is a polynomial of degree $d+1$. This contradicts the assumption that $f(X)$ is of degree at most $d$. Hence, $f(X)$ cannot have more than $d$ distinct roots.
\end{proof}

\begin{lemma} \label{SZ_Lemma}
    \textit{Schwartz-Zippel Lemma}: Let $f(X_1, X_2, \ldots, X_{\ell}) \in \F[X_1, X_2, \ldots, X_{\ell}]$ be a nonzero multivariate polynomial with total degree $d$. 
    If the variables $x_1, x_2, \ldots, x_{\ell}$ are chosen uniformly at random from $\F$, then
    \[
        \Pr\Bigl[f(x_1, x_2, \ldots, x_{\ell}) = 0\Bigr] \le \frac{d}{|\F|},
    \]
    where $|\F|$ denotes the size of the field.
\end{lemma}
The univariate case follows by setting $\ell = 1$.

\subsection{Zero Polynomial}
Consider a nonzero $\ell$-variate polynomial function $f$ of total degree $d$ over $\F_p$. 
For a randomly chosen point $r \in \F_p^{\ell}$, we have
\[
    \Pr[f(r) = 0] \le \frac{d}{|\F_p|}.
\]

For example, if $\F_p$ is such that $|\F_p| \approx 2^{256}$ and the total degree is $2^{20}$, then by Lemma~\ref{SZ_Lemma},
\[
    \Pr[f(r) = 0] \le \frac{2^{20}}{2^{256}} = \frac{1}{2^{236}},
\]
which is an exceedingly small probability.

Consequently, if for a random $r$ we find that $f(r)=0$, we can conclude—with overwhelming probability—
that $f$ is the zero polynomial. Although there is a slight chance of error, it is negligible in practice.

\subsection{Equality of Polynomial Functions}
Consider two multivariate polynomial functions $f(X)$ and $g(X)$, each having total degree at most $d$. 
By the Schwartz-Zippel Lemma, if a randomly chosen point $r$ satisfies $f(r) = g(r)$, then with high probability $f(X)$ and $g(X)$ are identical. To see this, define
\[
    h(X) = f(X) - g(X).
\]
Then $h(X)$ has degree at most $d$, and if $f(r) = g(r)$, we have $h(r)=0$. 
Since a nonzero polynomial of degree at most $d$ vanishes with probability at most $d/|\F|$, it follows that with high probability $h$ must be the zero polynomial. Hence, $f(X) = g(X)$.



\printbibliography[heading=subbibliography]